% !TEX encoding = UTF-8 Unicode
\documentclass[a4paper]{article}

\usepackage[english,serbianc]{babel}
\usepackage[unicode]{hyperref}
\hypersetup{colorlinks,
                   citecolor=green,
                   filecolor=green,
                   linkcolor=blue,
                   urlcolor=blue}

% Овако фонт неће бити блед тј. инсталираће
% се квалитетни фонтови за случај да већ нису
\usepackage{type1ec}

% Овако је могуће претраживати и копирати ћирилицу
% из генерисаног документа, независно од хифенације
\usepackage{cmap}
\defaulthyphenchar=127

% Овако је могуће у табели имати текст у два реда
\newcommand{\dvareda}[2][c]{\begin{tabular}[#1]{@{}c@{}}#2\end{tabular}}

% Овако је могуће табеларно представити ауторе
\newcommand{\autori}[8]{\author{\begin{tabular}{c c} \dvareda{#1\\\normalsize{#2}} & \dvareda{#3\\\normalsize{#4}}\medskip\\\dvareda{#5\\\normalsize{#6}} & \dvareda{#7\\\normalsize{#8}}\medskip \end{tabular}}}

% Овако је могуће имати кључне речи
\providecommand{\keywords}[1]
{
	\small	
	\textbf{\textit{Кључне речи ---}} #1
}

\begin{document}

\title{Трансформација петљи помоћу Кланга\\ \small{Семинарски рад у оквиру курса\\Конструкција компилатора\\Математички факултет, Београд}}

\autori{Јелена Јеремић}{mi16062@alas.math.rs}{Марија Ерић}{m17115@alas.math.rs}{Лазар Васовић}{mi16099@alas.math.rs}{Дарко Нешковић}{mi16208@alas.math.rs}

\date{23. јун 2020.}

\maketitle

\abstract{Размотрена је употреба Кланга као библиотеке у циљу трансформације свих петљи у \textit{C} коду у жељени тип (\textit{for}, \textit{while}, \textit{do-while}). На основу уводних разматрања имплементирана је апликација, у које сврхе је коришћен Клангов апликативни програмски интерфејс према апстрактном синтаксном стаблу. Саме измене вршене су у тексту кода. Описано је неколико изазова и проблема -- како успут решених, тако и отворених -- и сви су илустровани пратећим тест примерима.}

\keywords{Кланг (\textit{Clang}), \textit{AST}, језик \textit{C}, петље}

\tableofcontents

\newpage

\section{Увод}

Кланг је \cite{clang}...

\section{Апликација}

Алгоритам је...

\section{Проблеми}

Проблеми су...

\section{Закључак}

Закључак је...

\newpage
\addcontentsline{toc}{section}{Литература}
\appendix
\bibliography{Clang}
\bibliographystyle{unsrt}

\end{document}
